\documentclass[char]{guardians}
\begin{document}
\name{\cZeus{}}

You are \cZeus{}, leader of the mighty Greek Pantheon. Well, it would be mighty -- mightier -- if it weren't so small, and under the thumbs of the \cWarden{} and the \cCaretaker{}.  Many hundreds of years ago, you were surrounded by your extended family on Olympus and the only thing to worry about was how to hide your latest conquest from your wife. But now, most of your family is gone. Only your wife \cHera{}, daughter \cAthena{}, and son \cHephaestus{} remain. Oh, and \cPandora{}, that sweet girl who watches the box.

You aren't even on earth any more. For a few, glorious years, you and your family reigned over the earth from your seat on Mt. Olympus. Humans worshiped and feared you, and rightly so. It was a wonderful, carefree life -- appropriate for a God.  But one day all of Mt. Olympus was swallowed in a whirling vortex stronger than Charybdis. You and your family found yourselves ripped from earth and your followers, and imprisoned here in this stupid complex. Over the next 3000 years, the \cWarden{} and \cCaretaker{} attempted to brainwash your family into believing you were not in fact Gods. Instead, they claim you to be something called a ``celestial''. One by one, your family fell for their lies. They started to talk dreamily about the stars, and then the \cWarden{} and \cCaretaker{} would call a celebration. There would be a big party, and your beloved child or sibling would leave in a great ship and never return.

Much of your life now consists of sitting around the fake Mt. Olympus that your jailers constructed, and watching earth as though it were a television show. It is maddening to be unable to influence earth. Mortals don't even quake at the natural thunderstorms any more. Curse the \cWarden{} and \cCaretaker{}. No God should be more powerful than you. None should be able to contain you. And yet you are contained.

To try to distract you from your imprisonment the \cCaretaker{} concocted the idea of the \pGames{}. Once every hundred years or so, you are released from your prison cell -- alas, how could your beautiful Mt. Olympus have become no more than a prison cell? -- to compete with the Norse and Egyptian pantheons in a series of contests to prove dominance. Whatever else the \cCaretaker{} might be, at least \cCaretaker{\they} recognized that the Norse and Egyptians need to be regularly reminded that the Greek Pantheon is the strongest. Typically these games occur once every hundred years or so.

It has been 70 years since the last games.  Due to the arrival of some new God called \cUnity{}, from some new pantheon, the \cCaretaker{} has declared that the games should occur early, as a way to welcome \cUnity{\them}. Here is your chance to put into action a number of plans that you have been constructing for some time. Your primary objective is to restore the Greek Pantheon. In order to do this, you will have to accomplish several things in the next few hours while the \cCaretaker{} and \cWarden{} are distracted by the arrival of the new God.

When you were first brought to the complex, the pantheons of Greek, Norse, Egyptian and Shinto were all kept together. The other pantheons failed to accept the superiority of the Greek, and this led to substantial conflict, particularly between you and \cOdin{}, leader of the Norse, who would not concede mastery of the thunder and lightning to you. Eventually the \cCaretaker{} and \cWarden{} used the \stone, a relic of incredible power, to make facsimile of Mt. Olympus, Egypt, Asgard, and Takama to house each of the pantheons separately. If you could gain access to the \stone, you are confident you can use it to make a \emph{new} Mt. Olympus, beyond the reach of your jailers. \cHephaestus{} has long been consulting with you on the problem, and progress is painfully slow. 

You rarely have access to the \stone{} to examine it, and you had no hope of penetrating its shielding until now. Some disruption, possibly connected to the arrival of this new god, has weakened the shield around the \stone{} and \cHephaestus{} thinks that it may now be possible to breach the shields. The construction of a new Olympus will also apparently require an item from the Norse pantheon, \iHammer{\MYname} to be precise. How the forging of a new home for the Greeks could require a Norse relic is beyond you, but you trust your \cHephaestus{\offspring}'s conclusion.

Time is of the essence, so originally, you sent \cPandora{} to steal \iHammer{\MYname} as soon as you had access to the central courtyard. Unfortunately, \cPandora{\they} failed to acquire the item from \cHel{}. To your great consternation, \cHera{} came to you at almost the same time to explain that \cVal{} had tried to steal \iSandals{\MYname} from her. \cHera{} easily defended against the attempt of course, after all, \cHera{\they} is Greek -- but still, the implications are concerning. Are the Norse trying to escape control of the \cCaretaker{} and \cWarden{} too? Will they need the \stone? You won't stand for it if they are. The Greek are so much better than the Norse -- they cannot be allowed to show the Greek up in any way. You will prevail in your mission and they will fail.

While you worked out how to approach the situation given these recent developments, \cOsiris{} marched into the central courtyard and declared that the Egyptians would be putting up the \iNecro{\MYname} as a bet on who would win the games. The \iNecro{\MYname} is one of the most powerful Egyptian items and in order to avoid losing face, and conveniently enough to keep the \iSandals{\MYname} out of the Norse's hands, you put up the \iSandals{\MYname}. Perhaps not surprisingly, \cOdin{} promptly bet \iHammer{\MYname}. Now you have \emph{another} reason to win the games. You will need \iHammer{\MYname} to create a new Mt. Olympus, free from the control of the \cCaretaker{} and \cWarden{}.

Secondly, your \cAthena{\offspring} \cAthena{} has begun to long for the stars. With your family already so depleted, you simply cannot afford to lose \cAthena{\them}. There has to be a way to undo the brainwashing that the \cCaretaker{} has inflicted upon \cAthena{}. Perhaps something in the part of the complex reserved only for the \cCaretaker{} and \cWarden{}. You just have to break in... How convenient that those two are so distracted right now.

But even if you manage to keep what remains of your family here, you are greatly pained by the loss of the rest of your Pantheon.  \cHephaestus{} has been discussing the possibility of contacting earth, and you think that with minor modifications, you may be able to broadcast a message to the stars instead, recalling your family.

Oh,and there is \cPandora{}. \cPandora{\They} is always such an afterthought since \cPandora{\they} is just a human. Except \cPandora{} has \emph{the} box. About 100 years before your imprisonment, you encountered a glowing sphere of great power. Despite your every attempt to manipulate it or harness it's power, you could not. Fearing that another faction might master it's secrets instead, you had \cHephaestus{} construct a box to contain it. You then gave the box to \cPandora{} with the strict command to never open the box. So far, \cPandora{\they} have served you faithfully as an adopted \cPandora{\offspring}. You are concerned that someone within the complex knows what is in the box and may try to steal it, open it, and harness its power against you. The \cCaretaker{} and \cWarden{} have supposedly put some protection on \cPandora{} to prevent \cPandora{\them} from being killed, but given how much you trust them, it is of little comfort. You will have to keep an eye on \cPandora{} today, whenever possible.



\begin{itemz}[Goals]
  \item Ensure the Greeks win the \pGames{} by any means possible.
  \item Build New Mt. Olympus.
  \item Remove \cAthena{}'s yearning for the stars.
  \item Recall your wayward family to New Mt. Olympus.
  \item Show Greek superiority to the other Pantheons at all times.
  \item Keep an eye on \cPandora{}'s box and make sure no one finds out what is in it.
\end{itemz}


\begin{contacts}
  \contact{}
\end{contacts}

\end{document}
