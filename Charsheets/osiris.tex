\documentclass[char]{guardians}
\begin{document}

\name{\cOsiris{}}

You are \cOsiris{}, son of the Earth and the Sky. You once ruled over Egypt, during a time of prosperity that is always pleasant to remember. Then you were murdered by your brother \cSet{}, and your body was torn into fourteen pieces, scattered throughout the world. \cIsis{}, your wife, restored you to life, but not even a reincarnated god is allowed to return to the land of the living, so you remained in the underworld as its new ruler. You witnessed the judgement of countless thousands of the dead, and welcomed the just into your realm.

But that was millenia ago.

A vortex appeared in the land of the dead and engulfed you. You awoke with your brothers and sisters, and other gods and mortals unknown to you, imprisoned in a strangely featureless building. Two beings who referred to themselves as the \cCaretaker{} and the \cWarden{} claimed to be responsible for this. You were filled with outrage. Who were these beings, to treat you this way? But neither you nor any other imprisoned god could do anything. Whatever kind of gods the \cCaretaker{} and \cWarden{} are, they are far more powerful than any of you.

The \cCaretaker{} and \cWarden{} referred to you gods as ``celestials.'' They claim you are not really gods. But you know what you are, and you know the power you have wielded. Your jailers eventually conceded this, and even created a small realm for you to rule over, as well as separate realms for the other pantheons of gods they had brought here. Your ``Egypt'' was only a shadow of the Egypt you once ruled over, and you had only one mortal follower here, \cEgyptianHuman{}. You and your pantheon became restless from the confinement, and started to become more violent. The human \cEgyptianHuman{} only survived because of the protection provided to \cEgyptianHuman{\them} by the \cWarden{} and \cCaretaker{}.

In time, your jailers decided to hold a competition between the pantheons every hundred years. These Games are a diversion, but they make your existance tolerable. They brought you into contact with the gods of the other pantheons: Norse, Greek, and Shinto. These gods are strange and foreign to you, and you do not hold a high opinion of them. There are a small number of humans living with them, including one named \cJascha{} who your jailers have given the freedom to wander the pantheons even between Games.

After a millenium, some of your fellow gods and the gods of other pantheons became so homesick it started to affect their sanity. Horus was among the first of the Egyptian gods to suffer this. Oddly, he didn't appear to long for Egypt, but instead for some strange place among the stars which was certainly nowhere he had ever been before. Everyone who suffered this way was taken away by the \cCaretaker{} and \cWarden{}, and you could do nothing to stop them.

Over the centuries, more and more of your brothers and sisters were taken away, until only you, \cIsis{}, \cAnubis{}, \cSet{}, and the mortal \cEgyptianHuman{} remained. If only \cSet{}, and \cSet{} alone, could have been taken!

And, about five centuries ago, \cSet{} was the next Egyptian to lose his sanity. He too began to long for some strange place among the stars. However, \cSet{} was never taken away by your jailers. Before that happened, there was a massive explosion which rocked Egypt and, as you found out later, the realms of the other pantheons. \cSet{} was last seen near the center of the explosion; he didn't survive. You're sure \cIsis{} and \cAnubis{} suspect you had a hand in \cSet{}'s demise, but you know nothing of it.

Eventually the three of you gods who remained, along with \cEgyptianHuman{}, decided to escape the Complex. Although the \cWarden{} and \cCaretaker{} have powers far beyond yours, they are not omnipotent. \cSet{}'s death and the explosion which killed him clearly unsettled them. \cAnubis{} thought that \cSet{} was attempting to meddle with your jailers' spells of protection in a desperate attempt to get out of the Complex, and accidentally caused the explosion.

You had not made any real progress on figuring out how to escape when the most recent Games occurred seventy years ago. One event from those games is burned into your memory: your powers disappeared. You're not sure how it happened, but during the ``socialization time'' after the Games finished, your powers left you. You felt your powers becoming more and more distant. By the time the \cWarden{} and \cCaretaker{} made the gods return to their respective pantheons, your powers were so distant you couldn't feel them at all. You're certain your powers still exist in the Complex, somewhere, but you're not willing to tell anyone other than \cIsis{} and \cAnubis{} that you've lost them.

Over time your plans to escape have become more concrete. \cIsis{} brought the Norse goddess \cHel{} into your conspiracy. The others have come to the conclusion that \cSet{} has the key to defeating the security of the Complex, and want to resurrect him with the \iNecro{\MYname} the next time all of you meet at the Games. This is unacceptable. \cSet{} was not a friend to other gods even before he started losing his mind. You have not, however, been able to come up with a different plan. The loss of your powers makes this even more difficult: even if you could come up with a new plan of escape which doesn't involve \cSet{}, how are you to act on that plan while powerless?

Recently \cHera{}, a Greek goddess, sent you a coded message through \cJascha{}. She had some reason to want you to use your powers to help her the next time you meet at the Games, and wished to know what you might want in exchange. That much is obvious: you will need help finding your lost powers before you can use them to help her. You will never admit weakness to another pantheon, however; so you haven't replied or told \cHera{} what you want before you will help her. Ideally, you can get her help without ever telling her exactly what it is you're looking for.

In the past two days, your normal life in the Complex has been disrupted, and you've been presented with the best chance of escape in centuries. A new god called \cUnity{} was brought to the Complex, and your jailers decided to hold the Games immediately. In order to prevent \cIsis{} and \cAnubis{} from resurrecting \cSet{} before you could restore your powers, you offered the \cCaretaker{} the \iNecro{\MYname} as your pantheon's wager. You do want to win it back, of course, but for now it's safely out of the reach of anyone but your jailers. \cIsis{} and \cAnubis{} were not pleased with this, and think that your insistence on finding another way to escape is reckless\ldots{} as though resurrecting a dangerous, hostile god isn't!

Then, yesterday, there was a malfunction in the Complex. Parts of Egypt flickered in and out of existence for a few moments. Then, all went back to normal. Or so you thought, until you found \cEgyptianHuman{} crushed by a pillar which must have fallen when the block supporting it disappeared. \cAnubis{}, calm and silent as usual, simply extracted and embalmed \cEgyptianHuman{}'s heart, to be weighed against the \iFeather{\MYname}. \cIsis{} was heartbroken by \cEgyptianHuman{}'s death, and begged \cAnubis{} to have mercy and refrain from judging \cEgyptianHuman{\them}. Now you have no remaining worshippers. If someone is responsible for the death of \cEgyptianHuman{}, they owe you recompense for this.

This is all really starting to get to you. The sooner you get out of this place, the better.

\begin{itemz}[Notes]
  \item\textbf{Psychlim:} You may not admit that a god from another pantheon is better or more capable in any way than a god from the Egyptian pantheon, whatever you privately think. Feel free to come up with excuses, or to simply refrain from answering any questions which would require you to admit weakness.
\end{itemz}

\begin{itemz}[Goals]
  \item Find a way for your pantheon to escape the Complex without \cSet{}'s help. Take \cHel{} along if you can.
  \item Restore your lost powers.
  \item Regain the \iNecro{} for your pantheon, but don't allow \cIsis{} and \cAnubis{} to use it to resurrect \cSet{}.
  \item Find out what caused the event that killed \cEgyptianHuman{}, and get recompense if appropriate.
  \item Don't allow conflict over \cEgyptianHuman{} or \cSet{} to cause your pantheon to fall apart.
\end{itemz}

\begin{contacts}
  \contact{\cIsis{}} Your wife, the one person who has prevented you from losing your sanity in this prison. Her desire to raise \cSet{} is beginning to separate you.
  \contact{\cAnubis{}} Your fellow god of the afterlife, who judges the dead.
  \contact{\cHel{}} A Norse goddess who is in on your plans to escape.
  \contact{\cHera{}} A Greek goddess who is helping you restore your powers so that you can use your powers to help her.
  \contact{The \cCaretaker{}} One of the beings who is imprisoning you here.
  \contact{The \cWarden{}} The other being who is imprisoning you here.
\end{contacts}

\end{document}
