\documentclass[char]{guardians}
\begin{document}
\name{\cHera{}}

Most humans on earth know you only as the vindictive \cHera{\spouse} of \cZeus{}. A few -- your most loyal followers -- remember you for your primary role as guardian of mothers. You are the ultimate mother, and the Greek Pantheon is your family. Unfortunately, in recent years your family has dwindled to almost nothing. You have only your \cZeus{\spouse}, \cZeus{}; your \cAthena{\offspring}, \cAthena{}; your \cHephaestus{\offspring}, \cHephaestus{}; and your adopted \cPandora{\offspring}, \cPandora{} left. The five of you have been imprisoned on a false Mt. Olympus by two Gods from an unknown and incredibly powerful pantheon, the \cWarden{} and the \cCaretaker{}.

Once, you were surrounded by your whole extended family. You were the queen of Mt. Olympus and ruled over the earth with \cZeus{}. You were worshiped for your power, and your beauty was doubted by none save Paris. That was thousands of years ago. You were then ripped from your rightful place and imprisoned here in this complex, where you were forced to endure the lies of your jailers, who sought to delude you into believing you were not a goddess. One by one, your children and siblings fell for these lies and began to yearn for the stars instead of a return to your rightful place as gods and goddesses of Earth.  You miss your family dreadfully.

Eventually, you and \cZeus{} realized that it will be almost impossible to return to earth -- which is saying something for a Greek goddess to admit that anything might be impossible. You are proud, but not stupid. You recognize that the \cWarden{} and \cCaretaker{} are too powerful to be defied completely. So you cooked up a scheme to at least release your family from bondage to these two.

When you were first brought to the complex, the pantheons of Greek, Norse, Egyptian and Shinto were all kept together. The other pantheons failed to accept the superiority of the Greek, and this led to substantial conflict between the pantheons. Eventually the \cCaretaker{} and \cWarden{} used something called the \stone{} to make facsimile of Mt. Olympus, Egypt, Asgard, and Takama to house each of the pantheons separately. If you and \cZeus{} could gain access to the \stone{}, you should be able to find a way to manipulate it to create a \emph{new} Mt. Olympus, beyond the reach of your jailers.  \cZeus{} and \cHephaestus{} have long been working on the problem, and there have been many complications. It is your understanding that you will need access to the \stone{}, as well possession of \iHammer{\MYname}. Neither task will be easily accomplished.

The arrival of a new god has led to the opportunities you seek. With the \cCaretaker{} and the \cWarden{} are distracted, and the walls around the \stone{} weakened for some reason, \cZeus{} has initiated the final stage of the plan. In the midst of all the excitement, you were almost taken by surprise by \cVal{}. \cVal{} attempted to steal \iSandals{\MYname} from their display case on Mt. Olympus. Luckily you were in the antechamber, heard a commotion, and were able to stop \cVal{\them}. You immediately brought the news to \cZeus{}, who made the decision to give the \iSandals{\MYname} to the \cCaretaker{} as this year's bet for winning the \pGames{}. You certainly owe \cVal{} a rude surprise since \cVal{\they} seem to have forgotten that the Greek pantheon is not to be triffled with. It will be most appropriate for you to turn \cVal{\them} into a creature without wings -- how about a cat? -- for a while. Actually, anyone who annoys you sufficiently could be turned into a cat, but the spell is temporary, and you can only maintain it on one target at a time anyway. Probably best to save it for punishing \cVal{}

The \pGames{} are of course a centennial event, conceived by the \cCaretaker{} as a friendly competition between the pantheons. Every century, each pantheon that participates has wagered some item from their pantheon to make the competition more interesting. However, this year some incredibly powerful items have been wagered, including \iSandals{\MYname} and \iHammer{\MYname}.

Now you support your \cZeus{\spouse} in his endeavor to build a new Mt. Olympus, but after 3000 years of being treated as secondary, you are tired.  You would very much like to subvert the creation of  new Mt. Olympus to establish you as the ultimate ruler, and \cZeus{} as secondary.  To that end, you need some help. In particular, \cOsiris{} should be able to provide you with corrupted versions of the materials \cHephaestus{} has declared necessary for the rebuilding of Mt. Olympus. You took a great risk during the last interim between games, and sent a message to \cOsiris{}, requesting \cOsiris{\their} help. Despite the vagueness of your request, it was still incredibly dangerous. Should \cZeus{} find out\ldots{} A human named \cJascha{}, the only person except the \cCaretaker{} and \cWarden{} who can travel freely between the pantheons when they are locked, carried your coded message to \cOsiris{}. You can only hope \cOsiris{\they} received it and agree to help you.

Assuming \cOsiris{} agrees, you just have one more hurtle. Should \cZeus{} suspect sabotage, \cZeus{\they} may ask \cAthena{} to verify the sanctity of the items. This would expose your subterfuge. You therefore approached \cAthena{} obliquely about the problem, asking if \cAthena{\they} could think of any case in which subterfuge would be acceptable. \cAthena{} got super upset at you, and started yelling at you that you didn't love \cZeus{}. Where she gets these ideas mystifies you. \cAthena{\They} is like a teenager. For being a goddess of wisdom, she sure jumps to a lot of premature conclusions.

This disagreement rather predictably escalated into older disagreements, including \cAthena{}'s continued association with \cVal{} and \cFenrir{}. Especially with the recent incident with \cVal{}, you really can't understand how \cAthena{} can justify hanging out with them.  This fight finally devolved into \cAthena{} just yelling at you about how you don't want \cAthena{\them} to leave for the stars. (Of course you \emph{don't} want your beloved \cAthena{\offspring} to leave for the stars, but that's not the point.) In the end, \cAthena{} slammed the door to \cAthena{\them} room in your face. The two of you aren't speaking. You have no intention of speaking to \cAthena{\them} until \cAthena{\they} apologizes.

At least \cPandora{} is a dutiful \cPandora{\offspring}. \cPandora{\They} isn't really your daughter -- technically she is a human construct of \cAthena{}, \cHephaestus{} and \cZeus{} -- but you have grown to care for her over the millennia. \cPandora{} never contradicts you, \cPandora{\they} is always willing to help you out if you ask, and generally makes \cPandora{\them}self useful. \cPandora{\They} has this box though, and she refuses to let you see inside it. At first you were just mildly curious, but when you found out that \cZeus{} had given \cPandora{\them} the box with strict instructions never to open it, you became concerned. What could \cZeus{} want to hide from you? You don't want to upset \cPandora{}, but you must see what is inside \cPandora{\them} box.

Aside from the games, the need for the \stone{}, and some family drama with your \cAthena{\offspring}s, there are a few other things you would like to resolve. Although much of your family has fled to the stars, driven by a madness you hope never to experience, you are still a mother. It it breaks your heart that \cAmaterasu{} has been left all by \cAmaterasu{\them}self. All of the other Shinto gods have fled, and \cAmaterasu{\they} is all alone. You have been nagging at \cZeus{} since the last games to agree to adopt \cAmaterasu{}. You are not sure you are making much progress on this front yet, but you are determined. \cAmaterasu{} will of course also have to be persuaded that joining your family is preferable to being on \cAmaterasu{\them} own. This may also be tricky as Shinto gods are as proud as Greek gods.

Speaking of the other gods, while you might want to adopt \cAmaterasu{}, \cIsis{} is most emphatically not one you get along with. The rest of the Egyptian gods are generally courteous enough, but \cIsis{} has taken particular objection to the fact that married women are under your sphere of influence. For some reasons, \cIsis{\they} thinks that \cIsis{\they} deserves to be patron goddess of mothers and wives. \cIsis{\Them} name calling has pushed you to the limit, and covetous looks that \cZeus{} cast upon \cIsis{} during the last games was the last straw. While you might normally restrict yourself to turning \cIsis{\them} into an animal of some kind -- maybe a cow like you did Io -- \cIsis{} deserves worse. It is not normally possible to kill other gods, but you believe that the jewels present on certain particularly powerful relic such as \iSandals{\MYname} may hold the key. They clearly have enough power to kill a god, it need only be redirect -- if only their basic structure could be modified\ldots{}. Unfortunately, in the chaos after \cVal{} attempted to steal \iSandals{\MYname}, you forgot to remove the jewel before turning them over to \cCaretaker{}.


\begin{itemz}[Notes]
  \item\textbf{Ability:} You may turn another character into a cat by successfully using a waylay ability (3-count) on them; alternately you may target a willing or helpless character. This spell lasts for 5 minutes, and has a cool down of 20 minutes from casting time. For the duration of your spell, the target has been transformed into a cat (have them find a GM to acquire cat ears), and cannot talk or interact with other characters except as a cat. You may recall the spell at any time before the 5 minutes are up by finding your target and incanting ``release''.
  
\end{itemz}

\begin{itemz}[Goals]
  \item Win the \pGames{} to prove Greek superiority.
  \item Use the \stone{} to create new Mt. Olympus under your control. You will need \cOsiris{}'s help.
  \item Turn \cVal{} into a cat at least once.
  \item Kill \cIsis{} by modifying a powerful jewel to create a weapon capable of killing gods. %%mem packet on the molecular reasembler.
  \item Adopt \cAmaterasu{}. This will involve persuading \cZeus{} to accept her and \cAmaterasu{} to agree to go with you to new Mt. Olympus.
  \item Get \cAthena{} to apologize and reconcile with \cAthena{\them} -- after all, \cAthena{\they} is your \cAthena{\offspring}.
  \item Open \cPandora{}'s box to find out what is inside it.
\end{itemz}


\begin{contacts}
  \contact{\cZeus{}} Your \cZeus{\spouse} and the leader of the Greek Pantheon
  \contact{\cAthena{}} Your \cAthena{\offspring}, goddess of wisdom. The two of you are having a fight right now.
  \contact{\cPandora{}} Your human \cPandora{\offspring}. She is obsessed with her box.
  \contact{\cHephaestus{}} Your \cHephaestus{\offspring} who has been helping \cZeus{} with the \stone{}.
  \contact{\cOsiris{}} Leader of the Egyptian Pantheon. You have not yet had a chance to speak with \cOsiris{\them} in person since you sent \cOsiris{\them} the coded message requesting \cOsiris{\their} help.
\end{contacts}

\end{document}
