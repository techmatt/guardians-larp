\documentclass[char]{guardians}
\begin{document}

\name{\cUnity{}}

Two days ago, I was dead. Yesterday, I was \cUnity{\intro}, a god among mortals. Today, I'm in a prison inhabited by insane anachronisms and run by the malfunctioning constructs of an extinct species. I have every intention of getting out of this madhouse, but my window of opportunity to escape is frighteningly short. I barely know their names, but I'll have no choice but to try and work with these bizarre ``deities'' if I'm going to have any chance of breaking free.

My first moments of consciousness were confusing, to say the least. One moment I was a disembodied, unthinking speck, the next I was filled with the knowledge, ambition, thoughts, dreams, and troubles of a thousand humans. I knew how to prove the halting problem is undecidable, the various safety concerns that have to be taken into consideration when building a nuclear power plant, and the design schematics of the successor to the Tesla Model S.  My next few hours of existence were blissful. I had a perspective encompassing all of modern science, mathematics, and engineering, and I started connecting the dots. The breeder reactor design for thorium reactors can be optimized to surpass the efficiency of uranium fuel plants. I developed a lower-temperature synthesis pathway for sulfuric acid. And I came up with hundreds of experiments that I still intend to carry out on the drosophila melanogaster genome. While captivating, at some point I realized that these vast leaps of progress in modern science would go nowhere unless I started putting them into practice. So I decided it was time to look around and figure out who, what, and where I was.

I was pleasantly surprised to discover that I seemed to be inhabiting a human body with a perfectly functional set of limbs and senses. I even had clothes on, a marvelous development as there was honestly no time to waste on trying to purchase any. Taking stock of my surroundings, I deduced I was at the Akihabara district in Tokyo, certainly an acceptable location to start putting my plans into motion. Wandering into an electronics store, I gained access to the most marvelous collection of knowledge yet known, \emph{The Internet}. Realizing I could become lost on this network for years, I managed to summoned enough willpower to resist its call and focus my search on finding a suitable target for my plans. Weighing a number of factors, I eventually settled my sights on Kachicorp, a nearby robotics company  founded by \cKachiko{\intro}, a quite respectable roboticist. In the thirteen minutes it took me to navigate through the Tokyo Rail system to Kachicorp HQ, I had come up with three modifications to existing Kachicorp products and five new models. I estimated that within the week I would have displaced the existing CTO and laid the foundations for my plans to restructure the academic and industrial complexes of the world to suit my ever-growing plans for applied rationality, scientific progress, and eventual interstellar exploration and colonization. %Disappointingly, my timetable has been delayed by several hours by recent events but I have every intention of getting things back on track momentarily.

Upon arriving at Kachicorp, and assisted by the woefully inadequate security on their internal calendar system, I arranged a meeting with the CEO, \cKachiko{} within thirty minutes. Perhaps I should have found a more tactful way to approach the subject, as she seemed overly preoccupied with small details such as how I knew such intimate details of the schematics of Kachicorp. But once I got over these bumps and she realized the brilliance of my new designs, she lost herself in the potential of what I was describing. I found the resonance between myself and Kachiko intoxicating --- nothing is more rewarding than working with like-minded individuals to further our ambitions.

This is around the time things started to go wrong. While I was showing \cKachiko{} my design for a semi-automatic cooking system, a shimmering white vortex appeared in the room. Although I had no rational explanation for this phenomenon, I wasn't overly surprised; after all, my own existence seems to run quite contrary to the scientific consensus. Given that the vortex was rapidly sucking in all the objects in the room, I attempted to flee with no success. I waited until \cKachiko{} was pulled inside, hoping it might be sated upon acquiring a human consciousness, but as I expected the vortex was out for me alone and eventually I too was pulled through.

I awoke with \cKachiko{} in a city under construction. A strange humanoid entity walked up to us and introduced itself as the Caretaker. It claimed that I was an entity known as a celestial and that I present an unacceptable danger to the human population on Earth and consequently ``it was absolutely necessary'' for the \cCaretaker{} and its companion the \cWarden{} to transport me to their facility, the Celestial Containment Complex. They were building this city, Spire, as a place for me to ``feel comfortable'' during my stay here. While its architecture is indeed impressive it is also painfully apparent that I will not be allowed to advance any of my agendas here.

Once given a chance, I called up the myriad of documents the \cCaretaker{} was willing to provide. Apparently there are numerous other entities similar to myself trapped here, although they seem to be under the delusion that they represent deities from various time periods in human history. The largest pantheon seems to be Greek unsurprisingly led by Zeus, followed by Odin leading the Norse pantheon, an Egyptian pantheon with unclear leadership, and finally Amaterasu being the sole representative from the Shinto pantheon. While I was studying these documents, someone who appeared to be an ordinary human male came up to me and introduced \cJascha{\themself} as \cJascha{\intro} and then promptly left Spire. The \cWarden{} seemed unconcerned by \cJascha{}'s sudden arrival and departure, although I remained perplexed as the records indicate that under normal circumstances all prisoners are supposed to be detained to their individual realms.

The Caretaker, the Warden, and the complex itself all seem to be artifacts of a formerly galaxy-spanning race called the Guardians. In their vast wisdom they wandered the galaxy preventing \emph{dangerous beings} such as myself from interacting with innocent species such as humans, suppressing and killing evil species such as the \evilRace{}, and generally turning the entire galaxy into a micromanaged nanny-state. In their vast wisdom they also seem to have vanished 1000 years ago for unknown reasons, and the \cCaretaker{} and \cWarden{} are running off millennia-old instructions. While they are amazing artificially sentient beings, they appear to possess no semblance of free will, exhibiting only absolute loyalty to the Guardians and their ongoing command to contain and care for all celestials that attempt to interact with Earth. Although I loathe my containment, I also can't help but be excited by the fact that the Guardians are light-years ahead of human technology. I must learn everything I can about their technology, but the three areas that interest me most are the secrets of artificial sentience exemplified by my two jailers, the mechanisms that facilitate faster-than-light travel and communication, and the workings of the vortex that brought me here in the first place.

\cKachiko{} was understandably shaken by this rather abrupt turn of events, but she seems to be adjusting well. Once she realized that our jailers were not from a different species but instead just extremely advanced robots, like any self-respecting roboticist she rapidly shed her concerns and attempted to drill them on the details of their own cognition. They appear unwilling to yield anything of substance, but I'm certain she hasn't given up yet.

This cheerful welcome party came to an abrupt end once I managed to gain access to the files on how the complex works.  It turns out that the complex is located several miles beneath the surface of the moon, and opening the vortex from Earth creates a temporary hole in the bubble surrounding the complex. In four more hours, that hole will close and escaping after that time will be almost impossible. Realizing my timetable was now rapidly accelerated, once the constructs left to take care of other matters I seized control of Spire's central computer and modified its security firewall to exclude the \cCaretaker{} and \cWarden{}. I also exerted what control I could over the complex itself to prevent the bubble from closing and otherwise disrupt things in an attempt to throw the constructs off my trail. It is from within Spire that I will coordinate my escape, but it won't be easy. Although I haven't met with many of the others yet, I think it's safe to assume at least some of them aren't happy about the terms of their containment and will be willing to aid me in my plans to escape this prison. I'm not opposed to taking some of them with me if I can, but Kachiko definitely has the highest priority --- I'm both responsible for her being here and she seems to be one of the few people I can have a modern conversation with.

I've decided that once I get back to Earth and I have my empire in full-swing, it won't be any fun if I have no one at my side and have to do everything scientifically interesting myself. Fortunately, it seems as if the Guardians have a device called the \assembler{} that is exactly what I need to help \cKachiko{} (and perhaps others, if I find anyone worthy) exceed the limits of her mortal form. This device is able to fundamentally restructure entities, and while it can't make Kachiko into a being like the other gods here, it can certainly extend her lifespan along with her physical and mental abilities. The device also has a number of other functions that might be interesting to play with, although it is in a fairly secure area of the complex that I can't yet get into.

To build Spire, the constructs used a powerful artifact called the \stone{}. Although the \stone{} appears to be tightly integrated into the complex, the records the Guardians left on the construction of the complex say that the stone is just a bulky form of a more primal reactor called a \core{}, several of which rest at the heart of the facility and enables virtually lossless mass-energy conversion. This ingenious reactor is well beyond my ability to understand just yet, but assuming I manage to extract one I can do the most amazing things with it once I get back to Earth.

I've only just now wandered outside of Spire to the rest of this prison. Most of the deities seem to be busying themselves with some sort of tournament or game. I don't really know any of the details; apparently this is supposed to be a likely pointless team-building exercise orchestrated by the \cCaretaker{} to welcome me to the Complex.

%the AIs, once they get free will, can super-power themselves in the assembler

%Gain access to the central power core and use it to create a power source to take elsewhere.  This will likely mess up the AIs.


\begin{itemz}[Goals]
  \item Escape from the Complex with \cKachiko{} and possibly others before the bubble re-seals itself.
  \item Keep Kachiko safe.
  \item Mentor anyone who shows sufficient interest and aptitude in the ways of innovation and rationality, and possibly use the \assembler{} to help them along the path.
  \item Learn the secrets of Guardian technology for jump-starting the human techno-sphere once I'm back on Earth.
  \item Don't allow anything too terrible to escape the Complex --- I can't build my empire if the Earth is a wasteland.
  \item Take one of the \core{} reactors with me when I leave.
\end{itemz}


\begin{contacts}
  \contact{\cKachiko{}} A Japanese roboticist who founded Kachicorp and got sent through the vortex with me.
  \contact{The \cCaretaker{}} A sentient being created by the Guardians who claims to be in charge of taking care of the facilities and the celestials.
  \contact{The \cWarden{}} The counterpart to the \cCaretaker{}, in charge of security and preventing celestial escape.
  \contact{\cJascha{\intro}} A human who seems to have free reign of the Complex.
\end{contacts}

\end{document}
