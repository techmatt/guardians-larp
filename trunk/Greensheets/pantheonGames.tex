\documentclass[green]{guardians}
\begin{document}
\name{\gPantheonGames{}}

The Pantheon Games are a team-building exercise instigated by the Caretaker and used by the gods to prove superiority over the other pantheons. Each pantheon wagers items of power on the Game's outcome. The winning pantheon may choose two items (typically, they reclaim their own item and that of their greatest rival), while the second pantheon may claim the third remaining item. The gods and the Caretaker accept Amaterasu as the arbiter of the games, although ultimately it is the Caretaker who secures the items and oversees their transfer to the victorious pantheons.

With Unity's arrival, the Caretaker has stated that Unity will participate as part of the Egyptian pantheon even though \cUnity{\they} has only arrived a few hours ago. This iteration of the Games, three items of unusually strong power have been wagered:

\begin{itemize}
  \item \textbf{Greek Pantheon} --- \iSandals{}
  \item \textbf{Norse Pantheon} --- \iHammer{}
	\item \textbf{Egyptian Pantheon} --- \iNecro{}
\end{itemize}

The Games themselves are a series of five competitions that take place in \pAmphitheater{}. For each competition, each pantheon chooses a single champion to represent them. Each competition has the following structure:

\begin{itemize}
  \item Amaterasu or someone she or the Caretaker has appointed to arbitrate in her place must be present and announce the start of the competition.
	\item All the champions from each faction and at least one other observer must be present.
	\item When the competition is over, the arbiter announces the ranking of the pantheons. The winning pantheon receives 3 points, the second pantheon receives 2 points, and the remaining pantheon receives 1 point. If a pantheon's champion is not present for the competition, they receive nothing. There are no ties --- the arbiter always chooses a victor.
	\item The results of each competition are publicly recorded by the arbiter.
\end{itemize}
 
Each competition has a primary stat (the first letter of the competition name). \emph{This stat is correlated with victory but does not directly determine the outcome, and characters who know more details may not discuss these details in-game}. Some characters may be able to estimate a character's skill at a given competition.

Some characters may receive training in a particular skill during the course of play. This training takes the form of an out-of-game marble that will be given to you by your trainer. During the competition, you must proclaim the source of your skill as you transfer these tokens to the arbiter. Multiple skill tokens may be used for a single competition.

Some aspects may be invoked to enhance a competitor's skill. Aspects may be used on behalf of another player (for example, Zeus may invoke the Aspect of Lightning on Athena's sword). You must roleplay the way in which an aspect augments your character. You may attempt to gauge how effective a given aspect will be by discussing it with the arbiter beforehand.

\begin{itemize}
  \item \textbf{2:30 --- 3:00, Javelin Throw ($\mathbb{J}$ stat)} --- The champions hurl a javelin as far as possible down the courtyard. Both the distance and the style of the throw are important, with effects such as lightning and tornadoes being common spectacles.
	\item \textbf{3:00 --- 3:30, The Arts ($\mathbb{A}$ stat)} --- The champions exemplify the art and culture of their pantheon. The exact nature of the competition is decided individually by each pantheon, examples including dancing, singing, tailoring, craftmanship, and rhetoric. Pantheons may have multiple champions, but only if necessitated by the art style (ex. dancing).
	\item \textbf{3:30 --- 4:00, Mortal Combat (CR)} --- The champions engage in a three-way free-for-all gladiator style combat. Players may equip themselves with a weapon or shield from the equipment rack. The last conscious participant wins. The arbiter decides the outcome in the case where no combatants remain conscious. If multiple combatants are conscious, new items are chosen and the round repeats.
	\item \textbf{4:00 --- 4:30, Debate ($\mathbb{D}$ stat)} --- The champions each spend 60 seconds arguing the chosen topic. The prompt chosen by the Caretaker is ``Are gods and humans subject to the same code of morality?'', although Amaterasu has been known to choose a new topic of debate with little warning to participants.
	\item \textbf{4:30 --- 5:00, Monument Building ($\mathbb{M}$ stat)} --- Throughout the Games, the pantheons each amass materials (legos) for constructing a monument in their pantheon's image. These are built beforehand, but at this event each champion describes their pantheon's construction for judgment by Amaterasu.
\end{itemize}

The winning pantheons are announced by Amaterasu at the end of the Monument Building competition. In the case of a tie, a champion is chosen from each of the tied factions to engage in a duel. Once the final decision is announced, the Caretaker releases the chosen items to a member of the victorious pantheons.

\emph{These rules and times are dictated by the Caretaker, who also controls access to the wagered items. The gods have not always agreed to follow the rules or participate in the Games, although historically the Caretaker has ultimately accepted Amaterasu's decision regarding the victor of the Games.}

\end{document}
