\documentclass[green]{guardians}
\begin{document}
\name{\gArbiter{}}

Given the willful nature of the gods, arbitrating the games has always proven challenging. Over the years, a rough system has been established. You do not always follow this system to the letter, it is simply a framework you have found useful over the years. When things go wrong, you improvise.

For some competitions, such as Mortal Combat, arbitration is fairly easy as the outcome is visibly determined. For others, a mix of stats and personal judgement are combined. For these competitions, as each competitor performs, assign them a score between 1 and 5, with 1 being a dismal performance and 5 being perfect execution. Also use the use \aExamine{} ability on them to determine their stat for the competition. Use the sum of these two numbers to rank the competitors, breaking ties however you see fit.

Gods are busy people, and it can be difficult to gather the competitors together. You do your best by asking Odin, Zeus, and Osiris (although the Egyptians do not have a clear hierarchy that you can discern) beforehand who is competing in each competition, but when players do not show up for a game you simply assign them no points. Similarly, the times for the competitions is arbitrary and ultimately you often end up modifying the schedule as you see fit. When you feel the games are completed, you announce the score and rankings to the Caretaker.

The arbiter is also often busy. They may freely appoint others to arbitrate events in their place, including showing them this green sheet. For obvious reasons, only those not participating in the games are generally accepted as arbiters (\cAmaterasu{}, \cKachiko{}, \cJascha{}, \cCaretaker{}, \cWarden{}).

The winning pantheon selects two items first, then the second-place pantheon receives the remaining item. Only the Caretaker controls access to the wagered items. Arbitrating the games grants you no special abilities to access these items ahead of time.

\emph{Nothing in this document is mechanically necessitated. You are free to modify, change, or ignore the rules.}

\end{document}
