\documentclass[green]{guardians}
\begin{document}
\name{\gArbiter{}}

Given the willful nature of the gods, arbitrating the games has always proven challenging. Over the years, a rough system has been established. You do not always follow this system to the letter, it is simply a framework you have found useful. When things go wrong, you improvise.

For the Mortal Combat competition, arbitration is fairly easy as the outcome is visibly determined (when everyone knocks each other out, you judge based on style). For the others, a mix of stats and personal judgement are combined.

A competitior's score is the sum of the following three numbers:

\begin{itemize}
  \item \textbf{Base Stat} --- Use the \aExamine{} ability on the competitor to determine their stat for the competition.
  \item \textbf{Skill Tokens} --- Some characters may have skill tokens in the form of an out-of-game marble that represent recent training they have received for this event. Competitors must permanently surrender these tokens to you and describe who they received the tokens from. Each token provides a skill benefit of 2.
  \item \textbf{Your Personal Judgement and Aspects} --- As each competitor performs, judge if their performance is in some way truly exceptional, and assign them a number of bonus points between 0 and 4, with 0 being respectable and 4 being truly legendary --- 0 is a perfectly normal score to give unless you truly feel shocked by the quality of their performance. Characters may also embody certain Aspects during the course of their performance. These do not directly provide a bonus, but you are often impressed when they incorporate them into the performance and factor this into your judgement score.
	
\end{itemize}

Use the sum of these three numbers to rank the competitors, breaking ties however you see fit.

Gods are busy people, and it can be difficult to gather the competitors together. You do your best by asking Odin, Zeus, and Osiris (although the Egyptians do not have a clear hierarchy that you can discern) beforehand who is competing in each competition, but when players do not show up for a game you simply assign them no points. Similarly, the times for the competitions is arbitrary and ultimately you often end up modifying the schedule as you see fit, and trying to make the competitions move as quickly as possible. When you feel the games are completed, you announce the score and rankings to the Caretaker.

In the past, some gods have seen fit to modify the scoreboard while you are not watching. Unfortunately for them, you have a perfect memory.

You are also often busy. You may freely appoint others to arbitrate events in your place, including showing them this green sheet. For obvious reasons, only those not participating in the games are generally accepted as arbiters (\cAmaterasu{}, \cKachiko{}, \cJascha{}, the \cCaretaker{}, or the \cWarden{}).

The winning pantheon selects two items first, then the second-place pantheon receives the remaining item. Only the Caretaker controls access to the wagered items. Arbitrating the games grants you no special abilities to access these items ahead of time.

\emph{Nothing in this document is mechanically necessitated. You are free to modify, change, or ignore the rules. The arbiter may be bribed, choose randomly, lie about stats, or simply ignore this framework and personally judge each performance.}

\end{document}
